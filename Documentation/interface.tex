
\section{Definitions}
\archp and \archs exposes a simple interface for network debugging; see Table~\ref{tbl:interface}. We assume that each switch and host has a unique ID. We use the following definitions:
\begin{center}
\begin{itemize}[leftmargin=*, topsep=.3em, parsep=.2em]
\itemsep0em
  \item A {\tt linkID} is a pair of adjacent {\tt switchIDs} ($\langle \textrm{S}_\textrm{i}, \textrm{S}_\textrm{j} \rangle$);
  \item A {\tt Path} is a list of {\tt switchIDs} ($\langle \textrm{S}_\textrm{i}, \textrm{S}_\textrm{j}, \dots \rangle$);
  \item A {\tt flowID} is the usual $5$-tuple ($\langle$srcIP, dstIP, srcPort, dstPort, protocol$\rangle$);
  \item A {\tt Flow} is a ($\langle$flowID, Path$\rangle$) pair; this will be useful for cases when packets from the same {\tt flowID} may traverse along multiple paths.
  \item An {\tt epochID} represents an epoch in a switch.
  \item A {\tt pathInfo} is a list of ($\langle${\tt linkID}, a list of {\tt epochIDs}$\rangle$) pairs, where {\tt linkID} represents a link that a specific flow traverse and the corresponding {\tt epochIDs} at the first switch ($\textrm{S}_\textrm{i}$) in a {\tt linkID}.
  \item A {\tt flowStat} is a ($\langle$flowID, priority, byte counts$\rangle$) tuple.
  \item A {\tt timeRange} is a pair of {\tt timestamps} ($\langle \textrm{t}_\textrm{i}, \textrm{t}_\textrm{j} \rangle$); 
\end{itemize}
\end{center}
\archp and \archs support wildcard entries for {\tt switchIDs} and {\tt timestamps}. For instance, ($\langle \star, \textrm{S}_\textrm{j} \rangle$) is interpreted as all incoming links for switch $\textrm{S}_\textrm{j}$ and ($\langle \textrm{t}_\textrm{i}, \star \rangle$) is interpreted as ``since time $\textrm{t}_\textrm{i}$''.


\section{Interface}
Each host exposes the host API in Table~\ref{tbl:interface} and returns results for ``local'' flows, that is, for flows that have this host as their {\tt dstIP}. Similarly, each switch exposes the switch API and returns the hostIDs seen in an epoch. To collect the results distributed across individual end-hosts and switches, the controller may use the controller API --- to {\tt execute} a query at end-hosts or switches, to {\tt install} a query at end-hosts for periodic execution, or to {\tt uninstall} a query at end-hosts. For instance, to debug a network event, a controller use the controller API to locate, retrieve, and correlate the necessary data.
\begin{table}
\centering
{\small
\ra{1.15}
\begin{tabular}{@{}l@{}p{3.85in}@{}}
  \toprule
  \multicolumn{1}{c}{\textbf{Host API}} & \multicolumn{1}{c}{\textbf{Description}} \\
  \midrule
  \texttt{getFlows(linkID, timeRange)} & Return list of {\tt flows} that traverse {\tt linkID} during specified {\tt timeRange}. \\
  \texttt{getPaths(flowID, linkID, timeRange)} & Return list of {\tt Paths} that include {\tt linkID}, and are traversed by {\tt flowID} during specified {\tt timeRange}.\\
                                                 % which a specified link ID is
                                                 % present and are taken by a
                                                 % flow(s) with flow ID  within a
                                                 % given time range. \\

  \texttt{getPathInfo(flowID, timeRange)} & Return {\tt pathInfo} of {\tt flowID} in specified {\tt timeRange}. \\
  \texttt{getTelemetryData(linkID, epochID)} & Return list of {\tt flowStats} traversing the {\tt linkID} in an {\tt epochID}. \\

  \texttt{getCount(Flow, timeRange)} & Return packet and byte counts of a {\tt flow}
                                       within a specified {\tt timeRange}. \\
  \texttt{getDuration(Flow, timeRange)} & Return the duration of a {\tt flow} within
                                          a specified {\tt timeRange}. \\
  \texttt{getPoorTCPFlows(Threshold, timeRange)} & Return the {\tt flowIDs} for
                                        which {\tt protocol = TCP}, and the throughput in the current window is less than the threshold times previous window.  \\
  \texttt{getHighE2ELatencyFlows(delta)} & Return the {\tt flowIDs} whose inter-arrival packet time is greater than the delta. \\
  \texttt{getLossyFlows(Threshold)} & Return the {\tt flowIDs} whose {\tt protocol = TCP}, and the number of timeouts is greater than threshold times total number of packets received so far. \\
  \texttt{getAbnormalFlows()} & Return the {\tt flowIDs} whose packets arrive in out of order. \\
  \texttt{Alarm(flowID, Reason, pathInfo)} & Raise an alarm regarding
                                             {\tt Flow} with a reason code (e.g., TCP
                                             performance alert
                                             (\texttt{POOR\_PERF})), and the {\tt pathInfo} of {\tt Flow} \\
  \midrule
  \multicolumn{1}{c}{\textbf{Switch API}} & \multicolumn{1}{c}{\textbf{Description}} \\
  \midrule
  \texttt{getHosts(epochID)} & Return list of {\tt hostIDs} traverse a switch in the {\tt epochID}. \\
  \texttt{getHosts(timeRange)} & Return list of ($\langle${\tt epochID, hostIDs}$\rangle$) pairs, where the {\tt epochID} is in the {\tt timeRange}. \\
  \midrule
  \multicolumn{1}{c}{\textbf{Controller API}} & \multicolumn{1}{c}{\textbf{Description}} \\
  \midrule

 \texttt{execute(List$\langle$HostID/SwitchID$\rangle$,Query)} & Execute a {\tt Query} once at each host or switch specified in list of {\tt HostIDs} or {\tt SwitchIDs}; a {\tt Query} could be any of the ones from Host API or Switch API.\\
   % constructed with host
   %                                       APIs on a list of servers which are
   %                                       presented as a tree structure
   %                                       (\texttt{Tree}); \texttt{Agg}
   %                                       specifies a code that does multi-level
   %                                       data aggregation by referring to
   %                                       \texttt{Tree}. \\
  \texttt{install(List$\langle$HostID$\rangle$,Query,Period)} & Install a {\tt Query} at each host specified in list of {\tt HostIDs} to be executed at regular {\tt Periods}. If the {\tt Period} is not set, the query execution is triggered by a new event (\eg, receiving a packet).\\
                                                 % a list of servers in
                                                 % \texttt{Tree} and execute it at
                                                 % regular intervals
                                                 % (\texttt{Interval}); when an
                                                 % interval is not set, the query
                                                 % continuously processes online
                                                 % flow path records of newly
                                                 % arriving flows; note that
                                                 % \texttt{execQuery()} carries
                                                 % out one time execution of
                                                 % a query on demand. \\
  \texttt{uninstall(List$\langle$HostID$\rangle$,Query)} & Uninstall a {\tt Query} from each host specified in list of {\tt HostIDs}. \\


  \texttt{getLinkEpochStats(linkID, epochID)} & Return list of {\tt flowStats} traversing {\tt linkID} in the {\tt epochID}. \\
  \texttt{getContendingFlows(flowID, pathInfo)} &  Return list of {\tt flowStats}  contending with the {\tt flowID} across all links in the {\tt flowID} path. \\
 
  \bottomrule
\end{tabular}
}
\vspace{.05in}
\caption{Interface.}
\label{tbl:interface}
\end{table}

\begin{center}
\begin{lstlisting}[xleftmargin=.2\textwidth]
getLinkEpochStats(linkID, epochID):
  SwitchID = linkID[0]  # first switch in the linkID
  linkEpochStats = []
  hostIDs = execute([switchID], Query = getHosts(epochID))
  linkEpochStats = execute(hostIDs, Query = getTelemetryData(linkID, epochID))
  return linkEpochStats
\end{lstlisting}
\end{center}

\begin{center}
\begin{lstlisting}[xleftmargin=.2\textwidth]
getContendingFlows(flowID, pathInfo):
  contendingFlows=[]
  for linkID, epochIDs in pathInfo:
    for eID in epochIDs:
      linkEpochStats = getLinkEpochStats(linkID, eID)
  contendingFlows.append((linkID, eID, linkEpochStats))
  return contendingFlows
\end{lstlisting}
\end{center}
